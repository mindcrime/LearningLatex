% This is a simple sample document.  For more complicated documents take a look in the exercise tab. Note that everything that comes after a % symbol is treated as comment and ignored when the code is compiled.

\documentclass[fleqn]{article} % \documentclass{} is the first command in any LaTeX code.  It is used to define what kind of document you are creating such as an article or a book, and begins the document preamble
\usepackage[document]{ragged2e}

\usepackage{amsmath} % \usepackage is a command that allows you to add functionality to your LaTeX code

\title{Integration Table} % Sets article title
\author{Phillip Rhodes} % Sets authors name
\date{\today} % Sets date for date compiled

% The preamble ends with the command \begin{document}
\begin{document} % All begin commands must be paired with an end command somewhere
    \maketitle % creates title using information in preamble (title, author, date)
    
    \begin{equation}
      \int 1 \:dx = x + C
    \end{equation}

    \begin{equation}
      \int (x^r)\:dx =  \frac{ x^{(r+1)} }{(r+1)} + C
    \end{equation}
    
    \begin{equation}
      \int cos(x)\:dx = sin(x) + C
    \end{equation}

    \begin{equation}
      \int sin(x)\:dx = -cos(x) + C
    \end{equation}

    \begin{equation}
      \int sec^2(x)\:dx = tan(x) + C
    \end{equation}

    \begin{equation}
      \int csc^2(x)\:dx = -cot(x) + C
    \end{equation}

    \begin{equation}\
      \int sec(x)tan(x)\:dx = sec(x) + C
    \end{equation}

    \begin{equation}
      \int csc(x)cot(x)\:dx = -csc(x) + C
    \end{equation}



\end{document} % This is the end of the document